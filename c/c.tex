% A LaTeX template by Nehemjit Rai
\documentclass[12pt]{article}
\usepackage[left=25mm, top=1in, bottom=1in]{geometry}
\usepackage{graphicx}
\graphicspath{ {./ss} }
\usepackage{float} % for placing the picture correctly
\usepackage{verbatim}  % for multiline comments
\PassOptionsToPackage{hyphens}{url}\usepackage{hyperref} % text wrap for urls
\usepackage{listings} % for code snippets
\usepackage[linewidth=1pt]{mdframed} %for borders

\begin{document}
\title{C Cheat Sheet}
\author{Nehemjit Rai}
\maketitle

\pagebreak 
\tableofcontents
\pagebreak

\section{Hello World}
\begin{mdframed}
\lstinputlisting{code/hello.c}
\end{mdframed}

\section{Fundamentals}
\subsection{Variables}
\begin{mdframed}
\lstinputlisting{code/variables.c}
\end{mdframed}

\subsection{I/O}
\subsubsection{Basic I/O}
printf("control string",  variable1,  variable2, ...);\\
scanf("control string",\&pointer1,\&pointer2, ...);
\pagebreak

\subsubsection{Input using pointers}
\begin{mdframed}
\lstinputlisting{code/input.c}
\end{mdframed}

\subsubsection{More I/O}
\begin{mdframed}
\lstinputlisting{code/io.c}
\end{mdframed}
\pagebreak
Outputs:\\
(1)  10\\
(2)  50\\
(3)  x = 10\\
(4)  x = 10\\
input x: 40\\
(5)  x = 40\\
input x, y: 20 34.5\\
(6)  x = 20, y = 34.500000\\

\subsection{Conditional Operators}

\subsection{Loops}

\subsection{Statements}

\subsection{Notes}

\end{document}